\input{sem8/notes}


\documentclass[10pt,aspectratio=169]{beamer}

% Style
\usepackage{turnipmetropolis}

% TODOs
\usepackage[nofootnotes,warn]{turniptodo}

\usepackage{graphicx}

% Citations
\usepackage[beamer]{turnipcite}
\bibliography{2022_03_10}
% Numbers in beamer bibliography
\setbeamertemplate{bibliography item}{\insertbiblabel}

% For highlighting table rows
\usepackage{colortbl}

% For using OCR-d results table
\usepackage{amsmath}
\usepackage{amsfonts}
\usepackage{amssymb}
\usepackage{stmaryrd}
\usepackage{multirow}

% For centering too-wide table
\usepackage{adjustbox}

% For hiding columns in table
\usepackage{array}
\newcolumntype{H}{>{\setbox0=\hbox\bgroup}c<{\egroup}@{}}

% For strikethrough
\usepackage[normalem]{ulem}

% SI units
\usepackage{siunitx}

% Better tables
\usepackage{booktabs}

% Listings
\usepackage{listings}
\definecolor{mGreen}{rgb}{0,0.6,0}
\definecolor{mGray}{rgb}{0.5,0.5,0.5}
\definecolor{mPurple}{rgb}{0.58,0,0.82}
% \definecolor{backgroundColour}{rgb}{0.95,0.95,0.92}
\lstset{
  basicstyle=\ttfamily\small,
    %   backgroundcolor=\color{backgroundColour},   
    commentstyle=\color{mGreen},
    keywordstyle=\color{magenta},
    % numberstyle=\tiny\color{mGray},
    stringstyle=\color{mPurple},language=C,
  columns=fullflexible,
  escapeinside={/@}{@/}
}


% tikz options
\usepackage{tikz}
\usetikzlibrary{tikzmark}

\usepackage{etoolbox}



\newcommand{\code}[1]{\texttt{#1}}
\newcommand{\point}[1]{\item #1\vfill\null}
\newcommand{\spoint}[2]{\item #1\ifstrempty{#2}{}{\begin{itemize}
    \item #2
\end{itemize}}\vfill\null}
\newcommand{\sitem}[2]{\item #1\ifstrempty{#2}{}{\begin{itemize}
    \item #2
\end{itemize}}}

\newcommand{\twocol}[5][]{%
\begin{columns}\ifstrempty{#1}{}{[T]}
\column{#2\textwidth}
#4
\column{#3\textwidth}
#5
\end{columns}
}
\newcommand{\twostack}[2]{
#1


\vfill\null


#2
}


\newcommand{\citeauthoryear}[1]{\citeauthor{#1}, \citeyear{#1}}

\usepackage{turnipwordcount}

\title{\papertitle{}}
\subtitle{R01 Presentation/Report}
\date{\reportdate{}}
\author{Samuel Stark \emph{\href{mailto:sws35@cam.ac.uk}{(sws35)}}}
\institute{University of Cambridge}

% \setlength\itemsep{\fill}

\begin{document}

\maketitle
\note{
The handout notes collectively have each section of the review form.


\vspace{1em}
\begin{tabular}{l|c|r}
Section & Location & Word Count \\
\hline
    Summary & Slide \ref{00_summary} & \quickwordcount{00_summary} \\
    Pros and Cons & Slide \ref{01_pros_cons} & \quickwordcount{01_pros_cons} \\
    Motivation & Slide \ref{02_motivation} & \quickwordcount{02_motivation} \\
    Solution & Slide \ref{03_solution} & \quickwordcount{03_solution} \\
    Evaluation & Slide \ref{04_evaluation} & \quickwordcount{04_evaluation} \\
    Opinion & Slide \ref{05_opinion} & \quickwordcount{05_opinion} \\
    Questions & Slide \ref{06_questions} & \quickwordcount{06_questions} \\
\hline
    Total & & \quickwordcount{full}
\end{tabular}
}
\label{00_summary}



\metroset{sectionpage=none}
\section{Intro}
\metroset{sectionpage=progressbar}

\begin{frame}[standout]
\usebeamercolor[bg]{normal text}
    \begin{center}
        {\huge 
    Serval is a great project...
    
    
    \vspace{2em}
    
    
    \onslide<all:2>{...with a highly flawed writeup}
    }
    \end{center}
\note{
\textbf{Summary}

\input{\dirname/00_summary.tex}
}
\end{frame}

\begin{frame}{Background - Theorem Solving for Verification}
\begin{itemize}
    \point{So you want to formally verify your program against a specification}
    
    \spoint{Try \emph{interactive verification}}{
        Write a proof that the implementation satisfies the spec
        \item Used for CertiKOS\cite{guCertiKOSExtensibleArchitecture2016}
    }
    
    \spoint{Try \emph{auto-active verification}}{
        Derive proofs from annotated code (\cite{bondValeVerifyingHighperformance2017})
        \item Used for Komodo\cite{ferraiuoloKomodoUsingVerification2017}
    }

\end{itemize}

\note{CertiKOS = proof written and verified in Coq, the proof actually shows it supports concurrency?}
\note{Komodo = proofs deduced from annotations in assembly code (Vale language)}
\end{frame}


% -----------------------------------------
\begin{frame}{The Problem/Motivation - Prior Art}
\label{02_motivation}

\twocol{0.6}{0.4}{
\begin{itemize}

    \spoint{Even better, \emph{\alert{push-button verification}}}{
        Derive constraints on the final output from the program
        \item Solve the constraints and compare to the specification 
        \item Requires a ``finite interface''
        \item Used for Hyperkernel\cite{nelsonHyperkernelPushButtonVerification2017}
    }
    
    \spoint{Entirely automated process, much faster}{
        Manual proofs are \emph{long}
        \item CertiKOS = 200k lines of proof
        \item CertiKOS = 2 person-years
        \item Serval = 4 person-weeks
    }
\end{itemize}
}{
\begin{figure}
    \centering
    \includegraphics[width=\linewidth]{sem8/figs/push-button.png}
    \caption{Push-button workflow\cite{nelsonScalingSymbolicEvaluation2019a}}
\end{figure}
}

\note{
\footnotesize
\textbf{The Problem/Motivation}

\input{\dirname/02_motivation.tex}
}
\end{frame}


\begin{frame}{The Problem/Motivation - Rosette}
    % \twocol{0.5}{0.5}{
\begin{itemize}
    \point{Rosette lifts an \emph{interpreter} into a \alert{verifier}}
    \spoint{Deals in \emph{concrete values} and \emph{symbols}}{
        e.g. program instructions = concrete
        \item variables, arguments = symbols
    }
    \spoint{Rosette explores all possible program flows to find all possible end states}{
    Will merge equivalent states
    \item e.g. after branching}
\end{itemize}
    % }{
    % }
\end{frame}

\begin{frame}{The Problem/Motivation - Profiling}
    \twostack{
    \begin{itemize}
        \item Rosette authors later added a profiler to Rosette - SymPro\cite{bornholtFindingCodeThat2018}
        \item Shows where combinatorial explosion happens
    \end{itemize}
    }{
    \begin{figure}
        \centering
        \includegraphics[width=\linewidth,height=4cm,keepaspectratio]{sem8/figs/profiling.png}
        \caption{Example of profiling Rosette\cite{nelsonScalingSymbolicEvaluation2019a}}
    \end{figure}
    }
\end{frame}


\begin{frame}{The Problem/Motivation - Combinatorial Explosion}
    \twocol{0.5}{0.5}{
    \begin{itemize}
        \point{Rosette is designed to explore all execution paths}
        \point{Rosette can be overzealous if a value becomes a \emph{symbol}}
        \point{Rosette will try every potential symbol value, even if in context some are impossible}
    \end{itemize}
    }{
    \begin{center}
        \Large
        $PC \leftarrow $ if $X < 0$ {then} $4$ {else} $2$\\
        \vspace{2em}
        
        {Intuitively}\\
            $PC \in \{2, 4\}$\\

        \vspace{2em}
        {To Rosette}\\
            $PC \in \{1, 2, 3... \quad\includegraphics[height=1em,trim=0 2em 0 0,clip]{sem8/figs/amogus.png}...\quad{\LARGE \clubsuit}\}$\\


    \end{center}

    }
\end{frame}



% -----------------------------------------
\section{Serval's Solutions}
\label{03_solution}
\note{
\footnotesize
\textbf{Solution}
\input{\dirname/03_solution.tex}
}

\begin{frame}[standout]
\usebeamercolor[bg]{normal text}
    \begin{center}
        {\huge 
        Exploit domain knowledge to convert symbols into concrete values
    }
    \end{center}
\end{frame}

\begin{frame}{Symbolic Optimization - Program Counter}
    \twocol{0.5}{0.5}{
    \begin{itemize}
        \spoint{Program Counter}{
            Break the conditional \texttt{ite}
            \item Evaluate each concrete path separately
            \item Values must all be concrete
        }
        \point{Memory Addresses}
        \point{System Registers}
        \point{Monolithic Dispatch}
    \end{itemize}
    }{
    \begin{figure}
        \centering
        \includegraphics{sem8/figs/splic-pc.pdf}
        \caption{\texttt{split-pc}}
    \end{figure}
    }
\end{frame}

\begin{frame}{Symbolic Optimization - Memory Addresses}
    \twocol{0.5}{0.5}{
    \begin{itemize}
        \point{Program Counter}
        \spoint{Memory Addresses}{
            Memory is modelled in \alert{blocks}
            \item A symbolic address (e.g. integer-to-pointer converted) could point to \emph{any block}
            \item Concretize specific patterns
            \item Add a side condition to check the concretization's valid
        }
        \point{System Registers}
        \point{Monolithic Dispatch}
    \end{itemize}
    }{
        \begin{figure}
        \centering
        \includegraphics{sem8/figs/symbol-mem.pdf}
        \caption{Symbolic Memory Patterns}
    \end{figure}
    }
\end{frame}

\begin{frame}{Symbolic Optimization - System Registers}
    \begin{itemize}
        \point{Program Counter}
        \point{Memory Addresses}
        \spoint{System Registers}{
            Trap handlers rely on system register values
            \item Usually these are symbolic
            \item Some registers are only modified on boot
            \item Concretize only these registers
        }
        \point{Monolithic Dispatch}
    \end{itemize}
\end{frame}

\begin{frame}{Symbolic Optimization - System Call Dispatch}
    \twocol{0.5}{0.5}{
    \begin{itemize}
        \point{Program Counter}
        \point{Memory Addresses}
        \point{System Registers}
        \spoint{Monolithic Dispatch}{
        Many possible system calls
        \item All invoked through same trap
        \item Syscall number is symbolic -> all syscalls tried at once
        \item We know all valid concrete values
        \item Rewrite symbol as set of \texttt{ite}s
        }
    \end{itemize}
    }{
        \begin{figure}
        \centering
        \includegraphics{sem8/figs/split-case.pdf}
        \caption{\texttt{split-case}}
    \end{figure}
    }
\end{frame}

\begin{frame}{Symbolic Optimization - Mystery???}
    \begin{center}
        \emph{``It is known that compiler optimizations can increase the verification time on binaries [72].}\\
        \vspace{1em}
        
        \emph{To improve this, we continued to develop symbolic optimizations for Komodos.}\\
        \vspace{1em}

        
        \emph{Specifically, \alert{one new optimization} sufficed to reduce the verification time of Komodos
        for -O1 or -O2 to be
        close to that for -O0}\\
        \vspace{1em}
        
        
        \emph{(it did not impact the verification time
        of other systems).''}
    \vfill\null
    \onslide<2>{\Large Which one???}
    \end{center}
\end{frame}

\begin{frame}{Opinion}
    % \twocol{0.5}{0.5}{
    \vfill\null
    \begin{itemize}
        \spoint{Symbolic optimizations are great in concept}{
        Makes some previously-impossible programs possible to verify (Section 3.2)
        \item Where's the mystery optimization?}
        \spoint{Theoretically informed by profiling}{
        But ZERO proof of usefulness
        \item If you're making a whole new class of programs feasible, prove it!}
        \point{Weird that Rosette didn't do this before}
        \spoint{Weird they didn't choose to add it directly to Rosette}{
            Rosette authors implemented an automatic optimizer a year later - \cite{porncharoenwaseFixingCodeThat2020}
        }
        
    \end{itemize}
    % }{}
\end{frame}

% -----------------------------------------


% -----------------------------------------
\section{Serval's \sout{Evaluation} Case Studies}

\begin{frame}{Verifiers}
\twostack{
Four verifiers:
\begin{enumerate}
    \item RISC-V (rv64im + CSRs)
    \item x86-32 (Linux Kernel BPF JIT only)
    \item LLVM IR (subset derived from Hyperkernel = ``substantial subset''\cite{nelsonHyperkernelPushButtonVerification2017}?)
    \item BPF (essentially complete)
\end{enumerate}
    }{
    \begin{itemize}
        \item {Capabilities much more limited than one would expect based on abstract.}
        \sitem{Verifiers are more concise than previous work}{But we don't know what subsets they operate on}
        \sitem{``Wrote new interpreter tests and reused existing ones''}{No discussion of what tests passed/failed, which tests were actually used}
    \end{itemize}
    }
    
\end{frame}

\begin{frame}{Retrofitting: CertiKOS + Komodo}
\twostack{

\begin{columns}[T]
\column{0.5\textwidth}
\begin{block}{CertiKOS}
\begin{itemize}
    \item Ported from x86 to RISC-V
    \item Code left mostly unchanged
    \item Delegated ELF loading to each process
    \item Plugged security hole (found by Serval!)
\end{itemize}
\end{block}

\column{0.5\textwidth}
\begin{block}{Komodo}
\begin{itemize}
\item Ported from ARM to RISC-V
\item Reused unverified C code, not verified ASM
\end{itemize}
\end{block}
\end{columns}

}{

\begin{itemize}
    \item {Porting + verification took 4 person-weeks per system}
    \begin{itemize}
        \item {vs. person-years for CertiKOS}
        \end{itemize}
    \item {Symbolic optimizations were required to get them to verify at all}
    \item {Clearly, porting from other forms of verification is plausible}
    \begin{itemize}
        \item Although it does require slight modifications to proofs
    \end{itemize}
\end{itemize}

}

\end{frame}

\begin{frame}{Bugfixing: Keystone + BPF}

\twostack{
    \begin{columns}[T]
\column{0.5\textwidth}
\begin{block}{Keystone}
\begin{itemize}
    \item Wrote a specification by hand
    \item Compared specification to implementation by hand
    \item Found bug in implementation by hand
    \item Proved it was a bug with Serval
    \item Found undefined behaviour in binary with Serval
\end{itemize}
\end{block}

\column{0.5\textwidth}
\begin{block}{BPF}
\begin{itemize}
\item Transpiled Linux JIT compiler to Rosette
\item Combined BPF verifier with x86, RISC-V verifiers to compare side-effects of JITted instructions
\item Found 15 bugs using Serval
\item \cite{nelsonSpecificationVerificationField2020} expands on this, supports symbolic instructions
\end{itemize}
\end{block}
\end{columns}
}{
\begin{itemize}
    \item {Use of Serval is more limited, particularly for Keystone}
    \item {But Serval is shown to be flexible, and useful}
\end{itemize}
}
\end{frame}

\section{Conclusion}

\begin{frame}[t]{Pros and Cons}
\label{01_pros_cons}
\label{05_opinion}

\begin{columns}[T]
\column{0.5\textwidth}
\begin{block}{Pros}
\begin{itemize}
    \item \alert<all:2>{(Seemingly) well justified solutions}
    \item \alert<all:3>{Usefulness proven through case studies}
    \item \alert<all:4>{Excellent reproducibility}
\end{itemize}
\end{block}
\column{0.5\textwidth}
\begin{block}{Cons}
\begin{itemize}
    \item \alert<all:5>{No clear evaluation}
    \item \alert<all:6-8>{The paper puffs itself up dishonestly}
    \item \alert<all:9>{The paper is unfocused and overly long yet lacks essential detail}
\end{itemize}
\end{block}\end{columns}
% \twocol[NOT EMPTY]{0.5}{0.5}{

% }{

% }



% \textcolor{black}
\vfill\null
\begin{center}
    \only<all:2-4>{\Large  {\textbf{But}}
    \only<all:2>{{no concrete examples of profiling or improvements}}%
    \only<all:3>{{the case studies don't use Serval as much as I'd expect, and require lots of manual intervention}}%
    \only<all:4>{{\emph{\textless{}rebuttal not found\textgreater}}}
    }
    \only<all:6>{{\emph{Serval provides an extensible infrastructure for creating verifiers by lifting interpreters under symbolic evaluation, and a systematic approach to identifying and repairing verification performance bottlenecks using symbolic profiling and optimizations}}}
    \only<all:7>{{\emph{Serval provides an extensible infrastructure for creating verifiers \alert{using Rosette\cite{torlakGrowingSolveraidedLanguages2013}}, and a systematic approach to identifying and repairing verification performance bottlenecks using symbolic profiling and optimizations}}}
    \only<all:8>{{\emph{Serval provides an extensible infrastructure for creating verifiers \alert{using Rosette\cite{torlakGrowingSolveraidedLanguages2013}}, and a systematic approach to repairing verification performance bottlenecks using \alert{SymPro\cite{bornholtFindingCodeThat2018}} and symbolic optimizations}}}
    \only<all:9>{
    
    {\Large Examples: constant cross-reference marks,\\describing new optimizations in the case study section}
    
    
    
    {\Large Split it in two?}
    }
\end{center}
\vfill\null
% }

\note<handout:1>{
\footnotesize
\input{\dirname/01_pros_cons.tex}
}
\note<handout:2>{
\footnotesize
\textbf{Opinion}
\input{\dirname/05_opinion.tex}
}
\end{frame}

\begin{frame}{How to Improve?}
\begin{center}
\vfill\null    
    {\Large Split it in two?}
\vfill\null    
\begin{columns}[T]
\column{0.5\textwidth}
\begin{block}{Serval: Symbolic Optimizations}
    \begin{itemize}
        \item Explain optimizations in detail (Section~3)
        \item Prove they're useful
        \item Mention case studies as supporting evidence
    \end{itemize}
\end{block}
\column{0.5\textwidth}
\begin{block}{Serval: Case Studies in Verification}
    \begin{itemize}
        \item Simple optimization explanation (Section~2)
        \item Examine case studies in-depth (Section~6)
        \begin{itemize}
            \item e.g. steps taken to port CertiKOS and Komodo to RISC-V
        \end{itemize}
    \end{itemize}
\end{block}
\end{columns}
\vfill\null  
\end{center}
\end{frame}


\begin{frame}[standout]
\label{06_questions}
 \usebeamercolor[bg]{normal text}
    \begin{minipage}[c][0.5\paperheight]{\textwidth}
    \begin{center}
        {\Huge Questions?}
    \end{center}
    \end{minipage}

    \begin{center}
        \begin{itemize}
         \usebeamercolor[bg]{normal text}
            \item Why doesn't Rosette do these optimizations already?
            \item Why not add these optimizations to Rosette directly?
        \end{itemize}
    \end{center}

\note{
\footnotesize
\textbf{Questions for the Authors}

\input{\dirname/06_questions.tex}
}
\end{frame}

\appendix

% References slide
\turnipbibframe{}

% Appendix slides here

\end{document}